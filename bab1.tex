\chapter{PENDAHULUAN}
\section{Latar Belakang}
\label{sec:latarbelakang} %pelabelan ini opsional, biar bisa di klik kalo mengarahkan ke section tertentu.

Sampah merupakan salah satu penyebab terjadinya pencemaran lingkungan yang pada akhirnya menyebabkan kerusakan lingkungan. Pengelolaan sampah selama ini dilakukan secara konversional yaitu pengumpulan, pengangkutan dan pembuangan akhir di Tempat Pembuangan Akhir (TPA). Sampah perlu dikelola dengan cara memilih sampah antara organik dengan non organik, pendauran ulang dan pemanfaatan sampah, Adanya kendala dalam pengelolaan sampah yaitu rendahnya kesadaran masyarakat dan kurangnya pemahaman masyarakat tentang sampah. Solusi untuk mengatasi hal tersebut yaitu melalui pengembangan Aplikasi Bank Sampah agar sampah di masyarakat dapat di kelola dengan baik dan menumbuhkan kesadaran pada masayarakat itu sendiri dengan berpartisipasi menjadi nasabah pada Bank Sampah.     
Aplikasi Bank Sampah merupakan suatu sistem yang bertujuan untuk mengurangi populasi sampah pada masyarakat untuk dikelola dan dimanfaatkan sehingga menjadi sesuatu yang bermanfaat, dapat di daur ulang dan memiliki nilai ekonomi. Sumber tabungan bank sampah berasal dari masayarakat sekitar terlebih dahulu dan untuk jenis tabungan yang akan diterima oleh bank sampah adalah jenis sampah kering (Non Organik) diantaranya yaitu kertas, botol plastik, kaca, kardus, logam dan plastik. 
Berdasarkan uraian di atas, bahwa perlu dibangun sebuah aplikasi bank sampah yang dapat mempermudah admin, nasabaah(masyarakat) dan petugas kebersihan dalam melakukan pengelolaan bank sampah, proses penanbungan sampah pada bank sampah dan proses pengambilan sampah sehingga sampah-sampah dapat didaur ulang kembali dan menghasilkan nilai ekonomi.


\section{Permasalahan}
Dari latar belakang masalah dapat diangkat beberapa masalah antara lain :
\begin{enumerate}
\item Bagaimana proses admin dapat mengelola user pada bank sampah
\item Bagaimana proses masyarakat/nasabah menabung sampah pada bank sampah 
\item Bagaimana proses petugas sampah melakukan pengambilan sampah 
\end{enumerate}


\section{Tujuan}
\label{sec:tujuan}
Adapun perancangan desain web dan android ini yaitu :
\begin{enumerate}
\item Mempermudah admin dalam melakukan pengelolaan bank sampah 
\item Mempermudah nasabah/masyarakat dalam menabung sampah
\item Mempermudah petugas sampah dalam pengambilan sampah pada masyarakat/nasabah
\end{enumerate}

\section{Ruang Lingkup Studi}
Agar perancangan web dan android Bank Sampah ini  mudah  dimengerti tetapi tidak mengurangi tujuan penulis, maka penulis membatasi perancangan web tersebut sebagai berikut :
\begin{enumerate}
\item Sistem dirancang berbasis web khusus untuk admin.
\item Sistem dirancang berbasis android khusus untuk masyarakat/nasbah.
\item Menampilkan informasi jenis sampah apa aja yang akan di kelola beserta harga per kilogramnya.
\item Menerapkan system yang dapat terintegrasi antara web dengan android.
\end{enumerate}

\section{Sistematika Penulisan}
Laporan ini dibagi menjadi 5 bab yakni sebagai berikut.
\begin{enumerate}
\item Bab 1 Pendahuluan. Dalam bab ini dijelaskan mengenai latar belakang, permasalahan, tujuan, ruang lingkup dari penelitian, serta sistematika penulisan.
\item Bab 2 Teori Dasar. Dalam bab ini dijelaskan teori-teori dasar yang digunakan dalam melakukan penelitian seperti masalah banyak-partikel dalam mekanika kuantum, teori fungsional densitas, persamaan Kohn-Sham dan orbital.
\item Bab 3 Analisis dan Perancangan. Dalam bab ini membahas analisis sistem yang sedang berjalan dan sistem yang akan dibangun.  
\item Bab 4 Implementasi dan pengujian. Dalam bab ini dipaparkan mengenai implementasi sistem serta pengujian yang dilakukan.  
\item Bab 5 Kesimpulan dan Saran. Dalam bab ini diberikan kesimpulan yang diambil dari hasil penelitian dan saran-saran mengenai penelitian lebih lanjut yang dapat dilakukan.
\end{enumerate}


