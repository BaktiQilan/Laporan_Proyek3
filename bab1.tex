\chapter{PENDAHULUAN}
\section{Latar Belakang}
\label{sec:latarbelakang} %pelabelan ini opsional, biar bisa di klik kalo mengarahkan ke section tertentu.

Lorem ipsum dolor sit amet, consectetuer adipiscing elit. Etiam lobortis facilisis sem. Nullam nec mi et neque pharetra sollicitudin. Praesent imperdiet mi nec ante. Donec ullamcorper, felis non sodales commodo, lectus velit ultrices augue, a dignissim nibh lectus placerat pede. Vivamus nunc nunc, molestie ut, ultricies vel, semper in, velit. Ut porttitor. Praesent in sapien. Lorem ipsum dolor sit amet, consectetuer adipiscing elit. Duis fringilla tristique neque. Sed interdum libero ut metus. Pellentesque placerat. Nam rutrum augue a leo. Morbi sed elit sit amet ante lobortis sollicitudin. Praesent blandit blandit mauris. Praesent lectus tellus, aliquet aliquam, luctus a, egestas a, turpis. Mauris lacinia lorem sit amet ipsum. Nunc quis urna dictum turpis accumsan semper.


\section{Permasalahan}
Lorem ipsum dolor sit amet, consectetuer adipiscing elit. Etiam lobortis facilisis sem. Nullam nec mi et neque pharetra sollicitudin. Praesent imperdiet mi nec ante. Donec ullamcorper, felis non sodales commodo, lectus velit ultrices augue, a dignissim nibh lectus placerat pede. Vivamus nunc nunc, molestie ut, ultricies vel, semper in, velit. Ut porttitor. Praesent in sapien. Lorem ipsum dolor sit amet, consectetuer adipiscing elit. Duis fringilla tristique neque. Sed interdum libero ut metus. Pellentesque placerat. Nam rutrum augue a leo. Morbi sed elit sit amet ante lobortis sollicitudin. Praesent blandit blandit mauris. Praesent lectus tellus, aliquet aliquam, luctus a, egestas a, turpis. Mauris lacinia lorem sit amet ipsum. Nunc quis urna dictum turpis accumsan semper.


\section{Tujuan Penelitian}
\label{sec:tujuan}
Lorem ipsum dolor sit amet, consectetuer adipiscing elit. Etiam lobortis facilisis sem. Nullam nec mi et neque pharetra sollicitudin. Praesent imperdiet mi nec ante. Donec ullamcorper, felis non sodales commodo, lectus velit ultrices augue, a dignissim nibh lectus placerat pede. Vivamus nunc nunc, molestie ut, ultricies vel, semper in, velit. Ut porttitor. Praesent in sapien. Lorem ipsum dolor sit amet, consectetuer adipiscing elit. Duis fringilla tristique neque. Sed interdum libero ut metus. Pellentesque placerat. Nam rutrum augue a leo. Morbi sed elit sit amet ante lobortis sollicitudin. Praesent blandit blandit mauris. Praesent lectus tellus, aliquet aliquam, luctus a, egestas a, turpis. Mauris lacinia lorem sit amet ipsum. Nunc quis urna dictum turpis accumsan semper.


\section{Ruang Lingkup Studi}
Lorem ipsum dolor sit amet, consectetuer adipiscing elit. Etiam lobortis facilisis sem. Nullam nec mi et neque pharetra sollicitudin. Praesent imperdiet mi nec ante. Donec ullamcorper, felis non sodales commodo, lectus velit ultrices augue, a dignissim nibh lectus placerat pede. Vivamus nunc nunc, molestie ut, ultricies vel, semper in, velit. Ut porttitor. Praesent in sapien. Lorem ipsum dolor sit amet, consectetuer adipiscing elit. Duis fringilla tristique neque. Sed interdum libero ut metus. Pellentesque placerat. Nam rutrum augue a leo. Morbi sed elit sit amet ante lobortis sollicitudin. Praesent blandit blandit mauris. Praesent lectus tellus, aliquet aliquam, luctus a, egestas a, turpis. Mauris lacinia lorem sit amet ipsum. Nunc quis urna dictum turpis accumsan semper.



\section{Sistematika Penulisan}
Laporan ini dibagi menjadi 5 bab yakni sebagai berikut.
\begin{enumerate}
\item Bab 1 Pendahuluan. Dalam bab ini dijelaskan mengenai latar belakang, permasalahan, tujuan, ruang lingkup dari penelitian, serta sistematika penulisan.
\item Bab 2 Teori Dasar. Dalam bab ini dijelaskan teori-teori dasar yang digunakan dalam melakukan penelitian seperti masalah banyak-partikel dalam mekanika kuantum, teori fungsional densitas, persamaan Kohn-Sham dan orbital.
\item Bab 3 Analisis dan Perancangan. Dalam bab ini membahas analisis sistem yang sedang berjalan dan sistem yang akan dibangun.  
\item Bab 4 Implementasi dan pengujian. Dalam bab ini dipaparkan mengenai implementasi sistem serta pengujian yang dilakukan.  
\item Bab 5 Kesimpulan dan Saran. Dalam bab ini diberikan kesimpulan yang diambil dari hasil penelitian dan saran-saran mengenai penelitian lebih lanjut yang dapat dilakukan.
\end{enumerate}


