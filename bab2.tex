\chapter{LANDASAN TEORI}
\label{sec:landasan_teori}

\section{Tinjauan Pustaka}
\subsection{Aplikasi}
Aplikasi adalah program yang dibuat oleh pemakai yang ditujukan untuk melakukan suatu tugas khusus. Program dikelompokkan menjadi program aplikasi serbaguna dan program aplikasi spesifik.(Kadir,2002)
Pada pengertian umumnya, aplikasi adalah alat terapan yang difungsikan secara khusus dan terpadu sesuai kemampuan yang dimilikinya .Aplikasi merupakan suatu perangkat komputer yang siap pakai bagi user. Program aplikasi serbaguna adalah program aplikasi yang dapat digunakan oleh pemakai untuk melaksanakan hal-hal yang bersifat umum serta untuk mengotomasikan tugas-tugas individual yang bersifat berulang. Sedangkan program aplikasi spesifik adalah program yang ditujukan untuk menangani hal-hal yang sangat spesifik.

\subsection{Website}
Website adalah kumpulan halaman web yang saling terhubung dan filefilenya saling terkait. Web terdiri dari page atau halaman, dan kumpulan halaman yang dinamakan homepage. Homepage berada pada posisi teratas, dengan halaman-halaman terkait berada di bawahnya. Biasanya setiap halaman di bawah homepage disebut child page, yang berisi hyperlink ke halaman lain dalam web (Gregorius, 2000:30)

\subsection{Bank Sampah} 
Menurut (Dendawijaya, 2005) bank merupakan badan usaha yang mnghimpun dana dari masyarakat dalam bentuk simpanan dan disalurkan kepada masyarakat dalam berbagai bentuk seperti kredit atau bentuk lainnya dalam rangka meningkatkan taraf hidup orang banyak. Menurut (Slamet, 2002). BANK adalah "badan usaha yang menghimpun dana dari masyarakat dalam bentuk simpanan dan menyalurkannya kepada masyarakat dalam bentuk kredit dan atau bentuk-bentuk lainnya dalam rangka meningkatkan tarafhidup rakyat banyak".

Sampah adalah proses akhir kegiatan dalam kehidupan sehari-hari manusia atau proses alam dengan berbagai jenis baik padat, atau berupa zat organik atau anorganik yang dianggap sudah tidak berguna lagi dan dibuang lingkungan.

Sedangkan Bank Sampah adalah sebuah istilah untuk kegiatan pengelolaan sampah dimana masyarakat (disebut sebagai nasabah) menyetorkan sampah dimana sampah yang disetorkan tersebut masih layak. Bank sampah adalah suatu tempat yang digunakan untuk mengumpulkan sampah yang sudah dipilah-pilah. Sampah yang dikumpulkan adalah sampah yang mempunyai nilai ekonomis. Cara kerja bank sampah umumnya hampir sama dengan bank lainnya, ada nasabah, pencatatan pembukuan, dan manajemen pengelolaannya,Tujuan bank sampah selanjutnya adalah untuk menyadarkan masyarakat akan lingkungan yang sehat, rapi, dan bersih. Bank sampah juga didirikan untuk mengubah sampah menjadi sesuatu yang lebih berguna dalam masyarakat, misalnya untuk kerajinan dan pupuk yang memiliki nilai ekonomis.

\subsection{Android}
Android merupakan perangkat bergerak pada sistem operasi untuk telepon seluler yang berbasis linux.(Arifianto, 2011 : 1)`
Android merupakan OS (Operating System) Mobile yang tumbuh ditengah OS lainnya yang berkembang dewasa ini. OS lainnya seperti Windows Mobile, i-Phone OS, Symbian, dan masih banyak lagi. Akan tetapi, OS yang ada ini berjalan dengan memprioritaskan aplikasi inti yang dibangun sendiri tanpa melihat potensi yang cukup besar dari aplikasi pihak ketiga.

\subsection{Android Studio}
Android Studio adalah sebuah IDE untuk Android Development yang dikenalkan pihak google pada acara Google I/O di tahun 2013.
Android Studio merupakan suatu pengembangan dari Eclipse IDE, dan dibuat berdasarkan IDE Java populer, yaitu IntelliJ IDEA. Android Studio merupakan IDE resmi untuk pengembangan aplikasi Android.(Arifianto, 2011)

\subsection{Software Development Kit (SDK)}
Sebuah Software Development Kit (SDK atau devkit) tipikal merupakan satu set perkakas pengembangan software yang digunakan
untuk mengembangkan atau membuat aplikasi untuk paket software tertentu, software framework, hardware platform, sistem komputer, konsol video game, sistem operasi atau platform sejenis lainnya. Ia mencakup mulai dari pemrograman sederhana seperti sebuah application 22 programming interface (API), sampai dengan pemrograman yang lebih rumit dengan hardware yang canggih atau pada sistem embedded termasuk perangkat mobile.(Safaat (Latifah, 2011))

\subsection{CodeIgniter}
Menurut (Ardhana, 2013) CodeIgniter atau CI adalah framework PHP yang dapat digunakan untuk melakukan pengembangan dari suatu proyek pembuatan website agar dapat diselesaikan lebih cepat dibandingkan denganpembuatan website secara biasa.

\subsection{MySQL}
MySQL merupakan sebuah perangkat lunak atau software sistem manajemen basis data SQL atau DBMS Multithread dan Multiuser. MySQl sebenarnya merupakan turunan dari salah satu konsep utama dalam database untuk pemilihan atau seleksi dan pemasukan data yang
memungkinkan pengoperasian data dikerjakan secara mudah dan otomatis.

\subsection{XAMPP}
XAMPP adalah program aplikasi pengembang yang berguna untuk pengembangan website berbasis PHP dan MySQL. Versi terbaru program ini adalah XAMPP 1.7.7, yang dirilis pada tanggal 20 September 2011. Software XAMPP dibuat dan dikembangkan oleh Apache Friends. Perangkat lunak komputer ini memiliki kelebihan untuk bisa berperan 24 sebagai server web Apache untuk simulasi pengembangan website. Tool pengembangan web ini mendukung teknologi web populer seperti PHP, MySQL, dan Perl.


